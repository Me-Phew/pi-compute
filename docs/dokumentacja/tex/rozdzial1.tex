\newpage
\section{Ogólne określenie wymagań}		%1
Celem projektu jest stworzenie programu w języku C++, który umożliwi obliczenie przybliżonej wartości liczby \(\pi\) metodą numerycznego całkowania całki oznaczonej. Program powinien spełniać następujące wymagania:

\begin{itemize}
  \item Możliwość ustawienia przez użytkownika liczby podziałów przedziału całkowania, co wpływa na dokładność obliczeń.
  \item Obsługa wielowątkowości z możliwością ustawienia liczby wątków w celu zrównoleglenia obliczeń matematycznych.
  \item Wykorzystanie biblioteki \texttt{thread} zgodnej ze standardem POSIX.
  \item Wyświetlanie wyniku obliczonej całki oraz czasu wykonywania obliczeń na ekranie terminala.
\end{itemize}

Zakładamy, że wynikiem programu będzie poprawne obliczenie liczby \(\pi\) z dokładnością zależną od liczby podziałów przedziału oraz efektywne wykorzystanie dostępnych zasobów sprzętowych dzięki zastosowaniu wielowątkowości.

Przewidywane wyniki obejmują:
\begin{itemize}
  \item Skrócenie czasu obliczeń wraz ze wzrostem liczby wątków, co pozwoli zweryfikować poprawność implementacji wielowątkowości.
  \item Uzyskanie stabilnych wyników obliczeń \(\pi\) niezależnie od liczby wątków, przy założeniu wystarczającej liczby podziałów przedziału całkowania.
  \item Obserwację wpływu parametrów sprzętowych (liczba rdzeni, częstotliwość procesora) na wydajność programu.
\end{itemize}

Oczekujemy, że program pozwoli także na przeprowadzenie testów wydajnościowych i analizy wyników, które zostaną przedstawione w postaci wykresów i tabel. Wyniki te posłużą do oceny efektywności zastosowanych rozwiązań wielowątkowych oraz wyciągnięcia wniosków na temat wpływu ilości wątków na czas wykonania programu.
