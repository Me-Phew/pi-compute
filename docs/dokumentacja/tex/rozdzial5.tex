\newpage
\section{Wnioski}

Podczas realizacji projektu z wykorzystaniem narzędzi opartych na sztucznej inteligencji (AI) można sformułować następujące wnioski:

\subsection{Zagrożenia przy korzystaniu z AI}

Korzystanie z AI w programowaniu niesie ze sobą pewne ryzyko:
\begin{itemize}
  \item \textbf{Błędy w generowanym kodzie:} Kod generowany przez AI może zawierać błędy logiczne, syntaktyczne lub niezgodności z oczekiwaniami użytkownika, szczególnie gdy polecenia są niedokładne.
  \item \textbf{Niewłaściwe praktyki programistyczne:} AI czasami generuje kod, który nie spełnia standardów bezpieczeństwa, wydajności lub czytelności.
  \item \textbf{Zależność od AI:} Programiści mogą uzależnić się od narzędzi AI, tracąc umiejętność samodzielnego rozwiązywania problemów.
  \item \textbf{Ograniczona kontekstualność:} AI może nie znać pełnego kontekstu projektu, co prowadzi do błędnych założeń i nieoptymalnych rozwiązań.
\end{itemize}

\subsection{Czy należy korzystać z AI przy pisaniu programów?}

Korzystanie z AI przy pisaniu programów może być bardzo korzystne, pod warunkiem, że użytkownik jest świadomy jego ograniczeń. Narzędzia AI warto wykorzystywać do:
\begin{itemize}
  \item Automatyzacji powtarzalnych zadań.
  \item Szybkiego generowania szablonów kodu.
  \item Prototypowania rozwiązań i eksploracji różnych podejść.
\end{itemize}

Jednak należy pamiętać, że AI nie zastępuje wiedzy programistycznej, a jedynie ją uzupełnia.

\subsection{W czym AI pomaga, a w czym przeszkadza?}

\textbf{Pomoc:}
\begin{itemize}
  \item Przyspiesza pisanie kodu dzięki generowaniu gotowych fragmentów.
  \item Pomaga w nauce nowych technologii, sugerując rozwiązania na podstawie najlepszych praktyk.
  \item Wspiera w identyfikowaniu błędów i ich naprawianiu.
\end{itemize}

\textbf{Przeszkody:}
\begin{itemize}
  \item Może generować kod, który jest trudny do zrozumienia lub wymaga znacznych poprawek.
  \item Wymaga precyzyjnie zdefiniowanych poleceń, aby uniknąć niepożądanych rezultatów.
\end{itemize}

\subsection{Jak muszą być definiowane polecenia, aby AI poprawnie wygenerowała kod?}

Aby AI generowała poprawny i użyteczny kod, polecenia muszą być:
\begin{itemize}
  \item \textbf{Precyzyjne:} Opisujące dokładnie wymagania oraz oczekiwany efekt.
  \item \textbf{Kontekstowe:} Zawierające informacje o środowisku, języku programowania, używanych bibliotekach itp.
  \item \textbf{Krótkie, ale szczegółowe:} Zawierające konkretne wytyczne, np. „Napisz funkcję w języku C++, która oblicza całkę oznaczoną”.
\end{itemize}

\subsection{Czy może programować osoba nieznająca się na programowaniu?}

Na chwilę obecną, osoba bez wiedzy programistycznej może korzystać z AI, aby tworzyć prosty kod. Jednak:
\begin{itemize}
  \item Brak zrozumienia generowanego kodu może prowadzić do problemów z jego poprawianiem i debugowaniem.
  \item Bardziej zaawansowane projekty wymagają wiedzy technicznej, aby zrozumieć błędy i poprawnie je naprawić.
\end{itemize}

\subsection{Różnica między ChatGPT a GitHub Copilot}

\begin{itemize}
  \item \textbf{ChatGPT:}
        \begin{itemize}
          \item Generuje kod na podstawie naturalnych poleceń tekstowych.
          \item Nadaje się do wyjaśniania problemów, nauki i szerokich kontekstów programistycznych.
          \item Może działać poza edytorem kodu, oferując bardziej uniwersalne zastosowanie.
        \end{itemize}
  \item \textbf{GitHub Copilot:}
        \begin{itemize}
          \item Generuje kod w czasie rzeczywistym, bezpośrednio w edytorze.
          \item Koncentruje się na uzupełnianiu kodu i integracji z istniejącym projektem.
          \item Jest bardziej zoptymalizowany dla szybkiej pracy w środowisku IDE.
        \end{itemize}
\end{itemize}

Podsumowując, narzędzia AI, takie jak ChatGPT i GitHub Copilot, są wartościowymi wspomagaczami w procesie programowania, ale wymagają świadomego i krytycznego podejścia do ich wykorzystania.

\nocite{ChatGPT}
\nocite{MicrosoftExcel}
