\newpage
\section{Analiza problemu}

\subsection{Zastosowanie algorytmu}
Algorytmy całkowania numerycznego znajdują szerokie zastosowanie w wielu dziedzinach nauki i techniki. Są szczególnie użyteczne w sytuacjach, gdy funkcja nie posiada analitycznego rozwiązania całki lub gdy dostępne są tylko dane dyskretne. Przykłady zastosowań obejmują:
\begin{itemize}
  \item Obliczanie pól powierzchni pod wykresami funkcji w fizyce i inżynierii.
  \item Modelowanie zjawisk przyrodniczych, takich jak przepływ ciepła, dynamika płynów czy mechanika kwantowa.
  \item Analizę danych eksperymentalnych w statystyce i ekonomii.
  \item Symulacje komputerowe w informatyce i matematyce stosowanej.
\end{itemize}

W kontekście tego projektu, metoda numerycznego całkowania służy do aproksymacji wartości liczby $\pi$ na podstawie całki oznaczonej z funkcji $\frac{4}{1+x^2}$ na przedziale $[0, 1]$.

\subsection{Opis działania programu i algorytmu}
Program realizuje aproksymację liczby $\pi$ za pomocą numerycznej metody prostokątów. Algorytm podzielony jest na następujące kroki:
\begin{enumerate}
  \item Podział przedziału całkowania na określoną przez użytkownika liczbę podprzedziałów.
  \item Obliczanie wartości funkcji w środku każdego podprzedziału.
  \item Sumowanie wyników mnożonych przez szerokość podprzedziału, aby uzyskać przybliżenie całki.
  \item Zastosowanie wielowątkowości w celu równoległego obliczania sum częściowych, co przyspiesza obliczenia dla dużej liczby przedziałów.
\end{enumerate}

Wyniki obliczeń oraz czas wykonania są wyświetlane na konsoli oraz zapisywane do pliku CSV.

\subsection{Przykład zastosowania algorytmu}
Dla ilustracji, załóżmy, że całkujemy funkcję $\frac{4}{1+x^2}$ na przedziale $[0, 1]$ przy użyciu 4 podprzedziałów:
\begin{enumerate}
  \item Podzielmy przedział $[0, 1]$ na 4 równe części: $[0, 0.25]$, $[0.25, 0.5]$, $[0.5, 0.75]$, $[0.75, 1]$.
  \item Obliczmy wartości funkcji w środku każdego przedziału:
        \begin{align*}
          x_1 & = 0.125, & f(x_1) & = \frac{4}{1 + 0.125^2}, \\
          x_2 & = 0.375, & f(x_2) & = \frac{4}{1 + 0.375^2}, \\
          x_3 & = 0.625, & f(x_3) & = \frac{4}{1 + 0.625^2}, \\
          x_4 & = 0.875, & f(x_4) & = \frac{4}{1 + 0.875^2}.
        \end{align*}
  \item Wyznaczmy całkę jako sumę wartości funkcji pomnożoną przez szerokość przedziału $dx = 0.25$:
        \[
          \text{Całka} \approx dx \cdot \left(f(x_1) + f(x_2) + f(x_3) + f(x_4)\right).
        \]
\end{enumerate}
Przybliżona wartość całki odpowiada aproksymacji liczby $\pi$.

\subsection{Opis narzędzi użytych w projekcie}
\subsubsection{CMake i Ninja}
CMake został wykorzystany jako generator systemu budowania, ułatwiający zarządzanie zależnościami i konfiguracją projektu. Ninja, będący szybkim systemem budowania, został zastosowany w celu optymalizacji czasu kompilacji.

\subsubsection{Doxygen}
Doxygen umożliwił automatyczne generowanie dokumentacji kodu źródłowego w formie strony internetowej oraz plików PDF. Dzięki wykorzystaniu motywu Doxygen Awesome, strona dokumentacji zyskała nowoczesny i czytelny wygląd.

\subsubsection{Git i GitHub Actions}
System kontroli wersji Git zapewnił śledzenie zmian w kodzie oraz ich zarządzanie. GitHub Actions umożliwił automatyczne testowanie kodu, generowanie dokumentacji i tworzenie wydań projektu.

\subsubsection{Komponenty środowiska kompilacji}
Kompilacja projektu była przeprowadzana z użyciem GCC na systemie Linux oraz MinGW-w64 dla systemu Windows. Dostosowane środowiska MSYS2 (MinGW64/UCRT64) zapewniły kompatybilność z różnymi platformami.
